\subsection{Explain the term symbol (assuming L-S coupling) which describes the state of a multi-electron atom. Use an alkali atom to give an example.}


\paragraph{L-S kobling:} \ldots

Det totale impulsmoement bliver da $\Vec{J} = \Vec{L} + \Vec{S}$, hvor $\Vec{L} = \sum_i \Vec{l}_i$ er det totale baneimpulsmoment og $\Vec{S} = \sum_i \Vec{s}_i$ er det totale spin.

\paragraph{Termsymbol:} Termsymbolet\footnote{Dette er en fordansket betegnelse af det engelske \emph{term symbol}, men jeg er ikke bekendt med den danske betegnelse.} er en effektiv, kort og kompakt måde at skrive en stor del af den vigtigste information omkring et atom. For atomer, for hvilke L-S kolbingen gælder, kan termsymbolet skrives som
\begin{align}
    n^{2S+1}L_J \: ,
\end{align}
hvor $n$ er hovedkvantetallet, $2S+1$ er multipliciteten, altså antallet af $J$-tilstande, og
\begin{align}
    J = \abs{L - S}, \: \abs{L - S}, \: \ldots \: , \abs{L + S} - 1 \: , \abs{L + S} \: , 
\end{align}
er det totale impulsmoment. Både $n$, $S$ og $J$ skrives med deres respektive tal, mens $L$ skrives som den tilhørende orbital, se \cref{tab:Q14_ImpulsmomenterOrbitalerOgDeresNavne}.
\begin{table}[!h]
    \centering
    \begin{tabular}{|c|c|c|c|c|c|c|c|c|c|c|c|}
        \hline
        $L=$ & 0 & 1 & 2 & 3 & 4 & 5 & 6 & 7 & 8 & \ldots\\
        \hline
        \textbf{Orbital} & S & P & D & F & G & H & I & K & L & \ldots \\
        \hline
        \textbf{Navn} & Sharp & Principal & Diffuse & Fundamental & & & & & &\\
        \hline
    \end{tabular}
    \caption{Impulsmomenter med tilhørenden orbitaler og deres obitalnavne.}
    \label{tab:Q14_ImpulsmomenterOrbitalerOgDeresNavne}
\end{table}

Som eksempel kan man kigge på natrium ($Z=11$), hvilket er et alkalimetal, hvilket vil sige, at det kun har én enkelt valenselektron (elektron i yderste skal), hvilken er uparret. De to indre skaller ($n=1$ og $n=2$) er fyldt, og ved brug af Hunds regler\footnote{Hunds regler:
\begin{enumerate}
    \item For enhver atomgrundtilstand vil det totale elektronspin have den maksimale værdi tilladt ad Paulis eksklusionsprincip.
    \item For en given multiplicitet, og dermed for et bestemt spin $S$, vil den laveste energitilstand være svarende til termen med det største baneimpulsmoment $L$.
    \item For a given term, in an atom with outermost subshell half-filled or less, the level with the lowest value of the total angular momentum quantum number $J$ lies lowest in energy. If the outermost shell is more than halffilled, the level with the highest value of $J$ is lowest in energy.
\end{enumerate}
} kan vi bestemme, at den uparede elektron vil være i $3S$-orbitalen, og siden, at elektronen ligger i $S$-orbitalen, da må $L = 0$, og siden, at vi har tale om en enkelt elektron, da vil $S = 1/2$, så vi vil få $J = L + S = 0 + 1/2 = 1/2$. Derved bliver termsymbolet
\begin{align}
    n^{2S+1}L_J &= 3^{2 \cdot \frac{1}{2} + 1}S_\frac{1}{2} = 3^2S_\frac{1}{2} \: .
\end{align}
$ $\\\\

\ldots (Der sker vist et eller andet med ''In alkali metals, the atoms only have a single, unpaired electron in the outermost shell, the rest are filled (w.
exception of some of the outermost shells, due to the quantum defect in the central-field approximation).''



$ $\\\\
BØR DER SKRIVES MERE TIL DETTE SPØRGSMÅL???