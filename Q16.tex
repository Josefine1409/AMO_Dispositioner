\subsection{Outline the origin of the hyperfine structure. What are the assumptions made for the nucleus? Discuss the resulting splitting of the energy states.}


Ligesom elektronen har protonerne og neutronerne i kernen også et spin, da alle elementarpartikler har dette, og der tages nu højde for, at kernen også har et spin, kaldet \emph{kerneimpulsmoment} $I$ (eng. nuclear angular momentum). Som andre impulsmomenter giver kerneimpulsmomentet anledning til et magnetisk dipolmoment
\begin{align} \label{eq:Q16_NuclearAngularMomentumMagneticMoment}
	\vec{\mu}_I &= g_N \frac{\mu_N}{\hbar} \vec{I} \: ,
\end{align}
hvor der, modsat elektronens dipolmoment, ikke er et negativt fortegn, da dette stammer fra elektrones ladning\footnote{Hvis masse og ladning er fordelt identisk, så er faktoren $\gamma$ ($\vec{\mu} = \gamma \vec{L}$ for et impulsmoment $\vec{L}$) givet som $\frac{q}{2m}$, hvor $q$ er partiklens ladning og $m$ dens masse.}. I \cref{eq:Q16_NuclearAngularMomentumMagneticMoment} er $\mu_N = e\hbar/(Zm_p)$ kernemagnetronen (eng. nuclear magnetron). Kernen har et meget mindre dipolmoment end elektronerne, da kernemagnetronen er meget mindre end Bohrmagnetronen\footnote{Bohrmagnetronen er givet ved $\mu_B = \frac{e\hbar}{4m_e}$.}
\begin{align}
	\mu_B &= \frac{m_p}{m_e}\mu_N = 1836 \mu_N \: .
\end{align}

Af denne grund kan $H_{HFS}$ ses som en perturbation, og denne beregnes ved
\begin{align} \label{eq:Q16_HyperFineStructureHamiltonStart}
	H_{HFS} &= -\vec{\mu}_I \cdot \vec{B}_J = g_N \frac{\mu_N}{\hbar} \vec{I} B_J \frac{\vec{J}}{\abs{\vec{J} \: }} \: ,
\end{align}
hvor det er benyttet, at det magnetiske felt dannet af elektronen $\vec{B}_J = - B_J \vec{J}/|\vec{J}|$, hvor fortegnet kommer fra, at vi har beregnet det magnetiske felt ved kernes position, som er $-\vec{r}$ set fra elektronens synspunkt.