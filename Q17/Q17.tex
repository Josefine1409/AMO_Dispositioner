\subsection{Discuss the Zeeman Effect in the presence of hyperfine structure.}


\paragraph{Zeemaneffekten:}


\paragraph{Hyperfinstruktur:} Hyperfinstrukturen opstår, da man tager højde for kernens impulsmoment. Man bliver nødt til at skifte til at arbejde med det totale atomarimpulsmoment $\Vec{F} = \Vec{I} + \Vec{J}$, hvorved der skiftes basis fra $\ket{I,\: J,\: m_I,\: m_J}$ til $\ket{I,\: J,\: F,\: m_F}$. Energiskiftet grundet finstrukturen findes til at være
\begin{align}
    E_{HFS} &= \frac{A}{2}\left\{F(F+1) - I(I+1) - J(J+1)\right\} \: , \quad \text{med} \quad A = \frac{g_N \mu_N B_J}{\sqrt{J(J+1)}} \: ,
\end{align}
hvor $A$ er \emph{finstrukturkonstanten} og det resterende kaldes \emph{intervalfaktoren}.


\paragraph{Zeemaneffekten i hyperfinstruktur:} Som for Zeemaneffekten i finstruktur er der tre tilfælde: Hyperfinstrukturen regnes i et svagt magnetisk felt, et stærkt magnetisk felt, eller et felt med styrke midt i mellem.

\paragraph{Svagt magnetisk felt:} I det svage magnetiske felt vil energiskiftet grundet Zeemaneffekten være mindre end energiskiftet grundet hyperfinstrukturen, $E_{ZE} < E_{HFS}$, hvorfor Zeemaneffekten kan ses som en perturbation til hyperfinstrukturen
\begin{align}
    H_{ZE}
\end{align}

Dette kaldes den \emph{lineære Zeemaneffekt}.


\paragraph{}

Dette kaldes den \emph{ikke-lineære Zeemaneffekt} (eng. non-linear Zeeman effect)