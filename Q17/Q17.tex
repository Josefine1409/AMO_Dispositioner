\subsection{Discuss the Zeeman Effect in the presence of hyperfine structure.}


\paragraph{Zeemaneffekten:}


\paragraph{Hyperfinstruktur:} Hyperfinstrukturen opstår, da man tager højde for kernens impulsmoment. Man bliver nødt til at skifte til at arbejde med det totale atomarimpulsmoment $\Vec{F} = \Vec{I} + \Vec{J}$, hvorved der skiftes basis fra $\ket{I,\: J,\: m_I,\: m_J}$ til $\ket{I,\: J,\: F,\: m_F}$. Energiskiftet grundet finstrukturen findes til at være
\begin{align}
    E_{HFS} &= \frac{A}{2}\left\{F(F+1) - I(I+1) - J(J+1)\right\} \: , \quad \text{med} \quad A = \frac{g_N \mu_N B_J}{\sqrt{J(J+1)}} \: ,
\end{align}
hvor $A$ er \emph{finstrukturkonstanten} og det resterende kaldes \emph{intervalfaktoren}.


\paragraph{Zeemaneffekten i hyperfinstruktur:} Som for Zeemaneffekten i finstruktur er der tre tilfælde: Hyperfinstrukturen regnes i et svagt magnetisk felt, et stærkt magnetisk felt, eller et felt med styrke midt i mellem.

\paragraph{Svagt magnetisk felt:} I det svage magnetiske felt vil energiskiftet grundet Zeemaneffekten være mindre end energiskiftet grundet hyperfinstrukturen, $E_{ZE} < E_{HFS}$, hvorfor Zeemaneffekten kan ses som en perturbation til hyperfinstrukturen, så Hamiltonoperatoren bliver
\begin{align}
    H_{ZE} &= - \braket{\Vec{\mu}_F} \cdot \Vec{B} \: ,
\end{align}
hvor $\braket{\Vec{\mu}_F}$ er projektionen af $\Vec{\mu}_F$ ind langs $\Vec{F}$, hvilken skal benyttes, da denne er en middelværdi af $\Vec{\mu}_F$ ($\Vec{\mu}_F$ præciserer omkring $\braket{\Vec{\mu}_F}$), og $\braket{\Vec{\mu}_F}$ er velbeskrevet i basen $\ket{I \: J \: F \: M_F}$, hvilket $\Vec{\mu}_F$ ikke er. Denne basis benyttes idet, at vi arbejder med hyperfinstruktur for svage magnetfelter, hvorfor $\Vec{I}$ og $\Vec{J}$ kobler.

Det totale magnetiske dipolmoment for et atom er givet som summen af dipolmomentet grundet elektronens spin og baneimpulsmoment og grundet kerneimpulsmomentet
\begin{align}
    \Vec{\mu}_F &= \Vec{\mu}_I + \Vec{\mu}_J = - \frac{\mu_B}{\hbar} \left( g_L \Vec{L} + g_S \Vec{S} \right) \: ,
\end{align}
hvor $\mu_B = e\hbar/(2m_e)$ er Bohrmagnetronen, og $g_L \simeq 1$ og $g_S \simeq 2$ er Landé g-faktorer for hhv. baneimpulsmomentet og spinet.

AAAAAAAAAAAAAAAAAAAAAAAAA

%%%%%%%%%%%%%%%%%%%%


Det totale magnetiske dipolmoment for et atom er givet som summen af dipolmomentet grundet spin og grundet baneimpulsmomentet
\begin{align}
    \Vec{\mu}_J &= \Vec{\mu}_L + \Vec{\mu}_J = - \frac{\mu_B}{\hbar} \left( g_L \Vec{L} + g_S \Vec{S} \right) \: ,
\end{align}
hvor $\mu_B = e\hbar/(2m_e)$ er Bohrmagnetronen, og $g_L \simeq 1$ og $g_S \simeq 2$ er Landé g-faktorer for hhv. baneimpulsmomentet og spinet.

Et atoms interaktion med et eksternt magnetfelt beskrives ved Hamiltonoperatoren
\begin{align} \label{eq:Q15_StartudtrykForHamiltonForZeemaneffekten}
    H_{ZE} = -\braket{\Vec{\mu}_J} \cdot \Vec{B} \: ,
\end{align}
hvor $\braket{\Vec{\mu}_J}$ er projektionen af $\Vec{\mu}_J$ ind langs $\Vec{J}$ -- dette kan ses på \cref{fig:Q15_ZeemanEffectInThePresenceOfSpin}
-- hvilken skal benyttes, da denne er en middelværdi af $\Vec{\mu}_J$ ($\Vec{\mu}_J$ præciserer omkring $\braket{\Vec{\mu}_J}$), og $\braket{\Vec{\mu}_J}$ er velbeskrevet i basen $\ket{L \: S \: J \: M_J}$, hvilket $\Vec{\mu}_J$ ikke er. Denne basis benyttes idet, at vi arbejder med LS-kobling ($E_{ZE} \ll E_{SO}$), hvorfor også interaktionen i \cref{eq:Q15_StartudtrykForHamiltonForZeemaneffekten} kan ses som værende en perturbation til finstrukturen benyttende LS-kobling.

Projektionen $\braket{\Vec{\mu}_J}$ findes ved
\begin{align}
    \braket{\Vec{\mu}_J} &= \left(\Vec{\mu}_J \cdot \Hat{J}\right) \Hat{J} = \frac{\Vec{\mu}_J \cdot \Vec{J}}{\abs{\Vec{J} \,}^2} \Vec{J} = - \frac{\mu_B}{\hbar} \left( g_L \frac{\Vec{L} \cdot \Vec{J}}{\abs{\Vec{J} \,}^2} + g_S \frac{\Vec{S} \cdot \Vec{J}}{\abs{\Vec{J} \,}^2} \right) \Vec{J} \: ,
\end{align}
hvor $\Hat{J} = \Vec{J}/|\Vec{J}|$. Derved bliver Hamiltonoperatoren fra \cref{eq:Q15_StartudtrykForHamiltonForZeemaneffekten}, idet $\Vec{B} = B\Hat{z}$,
\begin{align}
    H_{ZE} &= -\left\{- \frac{\mu_B}{\hbar} \left( g_L \frac{\Vec{L} \cdot \Vec{J}}{\abs{\Vec{J} \,}^2} + g_S \frac{\Vec{S} \cdot \Vec{J}}{\abs{\Vec{J} \,}^2} \right) \Vec{J} \right\} \cdot \Vec{B} = \frac{\mu_B}{\hbar} \left( \frac{\Vec{L} \cdot \Vec{J}}{\abs{\Vec{J} \,}^2} + 2\frac{\Vec{S} \cdot \Vec{J}}{\abs{\Vec{J} \,}^2} \right) B J_z \: ,
\end{align}
hvor det er blevet brugt, at $g_L \simeq 1$ og $g_S \simeq 2$, hvormed energiskiftet grundet Zeemaneffekten bliver
\begin{align} \label{eq:Q15_EnergyZeemanInLSCouplingStart}
    E_{ZE} &= \braket{H_{ZE}} = \frac{\mu_B}{\hbar} \left( \frac{\braket{\Vec{L} \cdot \Vec{J}}}{\hbar^2 J(J+1)} + 2 \frac{\braket{\Vec{S} \cdot \Vec{J}}}{\hbar^2 J(J+1)} \right) B \braket{J_z} \: .
\end{align}

Forventningsværdien af $\braket{\Vec{L} \cdot \Vec{J}}$ og $\braket{\Vec{S} \cdot \Vec{J}}$ findes ved at $\Vec{J} = \Vec{L} + \Vec{S}$, så
\begin{align}
    \Vec{S} &= \Vec{J} - \Vec{L} \Rightarrow \Vec{S}^2 = \Vec{J}^2 + \Vec{L}^2 - 2\Vec{L} \cdot \Vec{J} \nonumber\\
    \Rightarrow \braket{\Vec{L} \cdot \Vec{J}} &= \frac{1}{2} \left(\braket{\Vec{J}^2} + \braket{\Vec{L}^2} - \braket{\Vec{S}^2}\right) = \frac{\hbar^2}{2} \left\{J(J+1) + L(L+1) - S(S+1)\right\} \: , \\
    \Vec{L} &= \Vec{J} - \Vec{S} \Rightarrow \Vec{L}^2 = \Vec{J}^2 + \Vec{S}^2 - 2\Vec{S} \cdot \Vec{J} \nonumber\\
    \Rightarrow \braket{\Vec{S} \cdot \Vec{J}} &= \frac{1}{2} \left(\braket{\Vec{J}^2} + \braket{\Vec{S}^2} - \braket{\Vec{L}^2}\right) = \frac{\hbar^2}{2} \left\{J(J+1) + S(S+1) - L(L+1)\right\} \: ,
\end{align}
hvorved vi får \cref{eq:Q15_EnergyZeemanInLSCouplingStart} til at blive
\begin{align}
    E_{ZE} &= \frac{\mu_B}{\hbar^3 J(J+1)} \frac{\hbar^2}{2}\left\{3J(J+1) - L(L+1) + S(S+1)\right\} B \braket{J_z} \nonumber\\
    &= \frac{\mu_B}{\hbar} \frac{3J(J+1) + S(S+1) - L(L+1)}{2J(J+1)} B \braket{J_z} \nonumber\\
    &= \frac{\mu_B}{\hbar} \frac{3}{2} + \frac{S(S+1) - L(L+1)}{2J(J+1)} B \hbar M_J \nonumber\\
    &= g_J M_J \mu_B B_z \: , \quad \text{hvor} \quad g_J = \frac{3}{2} + \frac{S(S+1) - L(L+1)}{2J(J+1)} \: .
\end{align}


%%%%%%%%%%%%%%%%%%%%

Dette kaldes den \emph{lineære Zeemaneffekt}.


\paragraph{}

Dette kaldes den \emph{ikke-lineære Zeemaneffekt} (eng. non-linear Zeeman effect), også kaldet Pashen-Backeffekten (eng. the Pashen-Back effect)