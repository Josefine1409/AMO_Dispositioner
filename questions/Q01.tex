\subsection{Tekst}

\emph{Briefly outline Bohr's model of the atom. What are the main problems of this model?}

Noter
- S.3-5.
- PP forelæsning 1
- Få ligninger, der stammer fra den klassiske mekanik.

Modellen
- Kvantiseret banebevægelse => Modellen kunne forklare de observerede spektre for hydrogen.
- Postulaterne
-- 1. Orbits as in classical mechanics
-- 2. Only particular orbits allowed
-- 3. Electron jumps between orbits
-- 4. Light energy h∙f = ΔE.

Problemer
- Eldyn: Ladede partikler i bevægelse udsender elektromagnetiske bølger. Bohr: Jeg antager, at de ikke udsender elektromagnetiske bølger. Data: Stemmer overens med Bohrs model.
-- Bohrs model fortæller ikke hvorfor, at de ikke udsender dem, men antager blot, at atomer ikke gør.
- Antager uendelig tung kerne, aka. stillestående kerne i centrum og kun bevægende elektroner i cirkelbaner.
-- Der skal tages højde for reduceret masse, hvilken er forskellig mellem atomers isotoper. Dette leder til en lille men aflæselig observerbar forskel i frekvensen af det udsendte lys fra forskellige isotoper i samme atom, hvilket kaldes isotopskift.