\subsection{Briefly outline Bohr's model of the atom. What are the main problems of this model?}

Bohrs atommodel bygger på den kendte viden fra Rutherfords eksperiment om, at atomer består af en lille og tung kerne med positiv ladning. Ser vi på hydrogen har vi en proton som kerne og en elektron kredsende omkring denne. Tiltrækningen mellem de to ladede partikler er givet ved Couloubkraften, hvilken er invers proportional med $r^2$ ligesom tyngdekraften er, hvorfor Bohrs atommodel bygger på den klassiske model for et tolegmesystem, f.eks. en planet omkring en stjerne, hvor banen er en cirkelbane, som fra klassiske mekanik opretholdes af en balance mellem centripetalkraften og Coulombbkraften
\begin{align} \label{eq:Q01_Centripetalkraft=Coulombkraft}
	\frac{m_e v^2}{r} &= F_\text{centripetal} = F_\text{Coulomb} = \frac{e^2}{4\pi\epsilon_0 r^2} \: ,
\end{align}
hvor $m_e$ er elektronens masse, $v$ er elektronens fart, $e$ er elementarladningen, hvilken er svarende til størrelsen af ladningen af en elektron eller proton, og $r$ er radius af banen.

Fra dette kan vinkelfrekvensen $\omega = v/r$ bestemmes
\begin{align}
    \frac{1}{r}\frac{m_e v^2}{r} &= \frac{m_e v^2}{r^2} = \frac{1}{r}\frac{e^2}{4\pi\epsilon_0 r^2} = \frac{e^2}{4\pi\epsilon_0 r^3} \nonumber\\
    \Rightarrow \omega^2 &= \frac{v^2}{r^2} = \frac{e^2}{4\pi\epsilon_0 m_e r^3} \: ,
\end{align}
hvilket er ækvivalent til Keplers lov, hvilket også var formodet, da der er benyttet de samme antagelser og atomet lige pt. er beskrevet fuldstændig klassisk.

Den totale energi i en sådan banebevægelse er givet ved summen af den kinetiske og potentielle energi
\begin{align}
    E &= T - U = \frac{1}{2}m_e v^2 - \frac{e^2}{4\pi\epsilon_0 r} \: ,
\end{align}
og ved at benytte \cref{eq:Q01_Centripetalkraft=Coulombkraft} fås den kinetiske energi til at være halvdelen af den potentielle energi
\begin{align}
    m_e v^2 &= \frac{e^2}{4\pi\epsilon_0 r}r = \frac{e^2}{4\pi\epsilon_0 r} \: ,
\end{align}
så den totale energi bliver
\begin{align} \label{eq:Q01_TotalenergiIHydrogenAtom}
    E &= - \frac{1}{2}\frac{e^2}{4\pi\epsilon_0 r} = - \frac{e^2}{8\pi\epsilon_0 r} \: ,
\end{align}
hvilken er negativ, da elektronen er i en bunden banebevægelse, så der skal tilføres energi til systemet for at fjerne elektronen fra banen.

Bohr formodede da, at kun baner med bestemte energier var tilladte, og at elektronen kun ændrer energi, når den hopper mellem baner, og denne overskudsenergi, når den hopper fra en yderlige bane til en inderligere bane, udsendes som lys af en bølgelængde bestemt ved energiforskellen mellem banerne.

\paragraph{Problem I -- Ingen stråling:} Bohr antager, at en ændring af energi kun kan finde sted, når elektronen hopper mellem de tilladte baner, men ud fra klassisk elektrodynamik udsender en ladet partikel i bevægelse energi i form af elektromagnetiske bølger (lys), men dette ville også betyde, at elektronen gradvist ville spiralere ind mod protonen og dermed også kunne forholde sig i mellemrummet mellem de tilladte baner. Dette stemmer ikke overens med Bohrs andre antagelser, hvorfor han antog -- da modellen ikke forklarer det -- at elektronen ikke radierer i dens banebevægelse, hvilket viser sig at holde stik med de eksperimentelle data.

Bohr postulerede, at impulsmomentet var kvantiseret,
\begin{align}
    m_e v r &= n\hbar \: , \quad \forall n\in \mathbb{N} \: ,
\end{align}
hvorfor energien ville blive kvantiseret.
Kombineres denne med \cref{eq:Q01_Centripetalkraft=Coulombkraft} fås radius som værende
\begin{align} \label{eq:Q01_RadiusSomFunktionAfN}
    \frac{m_e v^2}{r} &= \frac{e^2}{4\pi\epsilon_0 r^2} \nonumber\\
    \Rightarrow \frac{1}{r} &= \frac{e^2}{4\pi\epsilon_0 m_e v^2 r^2} = \frac{e^2 m_e}{4\pi\epsilon_0 m_e^2 v^2 r^2} \nonumber\\
    \Rightarrow r &= \frac{4\pi\epsilon_0 m_e^2 v^2 r^2}{e^2 m_e} = \frac{4\pi\epsilon_0 \hbar^2 n^2}{e^2 m_e} = \frac{\hbar^2 n^2}{\left(e^2/4\pi\epsilon_0\right) m_e} = a_0 n^2 \: ,
\end{align}
hvor $a_0$ er Bohrradien,
\begin{align}
    a_0 &= \frac{\hbar^2}{\left(e^2/4\pi\epsilon_0\right) m_e} \: .
\end{align}
Fra \cref{eq:Q01_TotalenergiIHydrogenAtom} og \cref{eq:Q01_RadiusSomFunktionAfN} kan man få Bohr formlen
\begin{align} \label{eq:Q01_BohrsFormel}
    E &= - \frac{e^2}{8\pi\epsilon_0 r} = - \frac{e^2}{8\pi\epsilon_0 a_0} \frac{1}{n^2} \: ,
\end{align}
hvor $n$ kaldes \textsf{hovedkvantetallet} (eng. principal quantum number).
Bohrs formel forudsiger, at lyset udsendt fra overgange, $n' \rightarrow n$, mellem de tilladte baner i atomet vil have bølgetal ($\tilde{\nu} = 1/\lambda$)
\begin{align} \label{eq:Q01_BoelgetalFraBohrsFormel}
    \tilde{\nu} &= R_\infty \left(\frac{1}{n^2} - \frac{1}{n'^2}\right) \: .
\end{align}
Udregnes de mulige bølgetal for hydrogen vil disse stemme forholdsvis overens med de udregnede bølgetal fra Rydbergs formel,
\begin{align}
    \tilde{\nu} &= \frac{1}{\lambda} = R \left(\frac{1}{n^2} - \frac{1}{n'^2}\right) \: .
\end{align}
Rydbergkonstanten $R_\infty$ i \cref{eq:Q01_BoelgetalFraBohrsFormel} er defineret som
\begin{align}
    hcR_\infty &= \frac{\left(e^2/4\pi\epsilon_0\right)^2 m_e}{2\hbar^2} \: ,
\end{align}
hvor $hc$ er en konversionsfaktor mellem energi og kvantetal, siden $R_\infty$ er givet i enheder af $\si{\per\metre}$ (eller $\si{\per\centi\metre}$, hvilket normalt benyttes). I laboratoriet har man ved hjælp af lasere kunnet finde en utrolig præcis værdi for Rydbergkonstanten $R_\infty = \SI{10973731.568525}{\per\metre}$.

\paragraph{Problem II -- Antagelse om uendelig tung kerne:} Selvom Bohrs formel giver en Rydbergkonstant $R_\infty$ meget tæt på værdien af Rydbergkonstanten for et hydrogenatom $R_H$, så er der et problem med den, hvilket kommer af, at Bohr har antaget en fikseret kerne, altså en uendelig tung kerne (hvorfor det nedsænkede $\infty$). I realiteten bevæger både protonen og elektronen rundt om deres samlede massemidtpunkt (center of mass). For en kerne med en endelig masse $M$, da kan ligningerne modificeres ved at udskifte elektronens masse $m_e$ med den reducerede masse
\begin{align}
    m &= \frac{m_e M}{m_e + M} \: ,
\end{align}
hvilket for hydrogen giver Rydbergkonstanten
\begin{align}
    R_H &= R\infty \frac{M_p}{m_e + M_p} \simeq R_\infty \left(1 - \frac{m_e}{M_p}\right) \: ,
\end{align}
hvor elektronmasse-til-protonmasse-forholdet er $m_e/M_p \simeq 1/1836$. Dette forhold er forskelligt for isotoperne af et grundstof, f.eks. hydrogen og deuterium, hvilket giver små men observerbare forskelle i frekvensen af det udsendte lys for forskellige isotoper af et grundstof, hvilket kaldes isotopskift.


\paragraph{Problem III -- Mangel på spin:} \ldots