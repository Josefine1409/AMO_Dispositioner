\section{Briefly outline Bohr's model of the atom. What are the main problems of this model?}
\sectionmark{Bohrs atommodel}

\noindent
\large
Sammenligning med solsystem grundet $F_\text{Coulomb} \propto 1/r^2$.\\
Elektron i baner givet ved $F_\text{centripetal} = F_\text{Coulomb}$
\begin{align*}
    \frac{m_e v^2}{r} &= \frac{e^2}{4\pi\epsilon_0 r^2} \: .
\end{align*}
Den totale energi i systemet
\begin{align*}
    E &= T + U = \frac{1}{2}m_e v^2 - \frac{e^2}{4\pi\epsilon_0 r} = - \frac{e^2}{8\pi\epsilon_0 r} \: .
\end{align*}
Postulater fra Bohr:
\begin{itemize}
    \item Kvantiserede baner: Kun baner med bestemte energier er tilladte.
    \item Energitab fra elektronen foregår kun ved kvantespring mellem de tilladte baner.
\end{itemize}
Problem I: Ingen stråling fra elektroner
\begin{itemize}
    \item Elektrodynamik: Accelererende ladede partikler udsender elektromagnetisk stråling (lys).
    \item Bohr antog, at dette ikke skete, da ellers ville elektronerne jo kunne befinde sig mellem banerne grundet et energiniveau mellem det for de tilladte baner.
    \item Bohrs antagelse stemmer overens med eksperimenter.
\end{itemize}
Find radius fra antagelse om kvantiseret impulsmoment ($m_e v r = n\hbar$)
\begin{align*}
    n\hbar &= m_e r \sqrt{\frac{e^2 r}{4\pi\epsilon_0 m_e r^2}} = m_e r \sqrt{\frac{e^2}{4\pi\epsilon_0 m_e r}} \\
    \Rightarrow r &= \frac{4\pi\epsilon_0 \hbar^2}{e^2 m_e}n^2 = a_0 n^2 \: .
\end{align*}
Bohrs formel bliver derved (ved indsættelse af $r$ i $E$)
\begin{align*}
    E_n &= - \frac{e^2}{4\pi\epsilon_0 r} = - \frac{e^2}{4\pi\epsilon_0 a_0} \frac{1}{n^2} = \frac{E_1}{n^2}
\end{align*}
Bølgetal fra Bohrs formel
\begin{align*}
    \Tilde{\nu} &= R_\infty \left(\frac{1}{n^2} - \frac{1}{(n')^2}\right) \: .
\end{align*}
Bohrs fundne Rydbergkonstant stemmer stort set overens med den fra Rydbers formel, men der er en lille afvigelse
\begin{itemize}
    \item Problem II: Antagelse om uendelig tung kerne
    \begin{itemize}
        \item Løsning ved reduceret masse i stedet for $m_e$, hvilket giver ''Rydberkonstantforholdet''
        \begin{align*}
            R_H = R_\infty \frac{m_p}{m_e + m_p} \simeq R_\infty \left(1 - \frac{m_e}{m_p}\right) \: , \quad \text{hvor} \quad \frac{m_e}{m_p} = \frac{1}{1836} \: .
        \end{align*}
        \item Isotopskift: Frekvensforskel i det udsendte lys fra forskellige isotoper grundet deres forskellige kernemasser.
    \end{itemize}
\end{itemize}
Problem III: Diverse ikkemedtagne effekter
\begin{itemize}
    \item Medtager ikke spin $\Rightarrow$ Ingen forklaring af fin- og hyperfinstruktur.
    \item Medtager ikke diverser relativistiske effekter.
    \item Overholder ikke Heisenbergs usikkerhedsprincip, da vi antager at kende banen helt præcis.
\end{itemize}
\normalsize