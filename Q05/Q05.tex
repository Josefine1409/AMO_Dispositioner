\subsection{Briefly outline the solution of the Schrödinger equation for hydrogen and comment on the result. What are the separation constants and why is the energy degenerate in $l$ and $m_l$?}


Hydrogen er det simpleste atom, og det eneste atom, som kan løses analytisk ved brug af Schrödingerligningen. Løsningen af hydrogenatomer kan overføres til andre atomer med samme egenskaber, altså alkalimetaller, som også kun har én enkelt valenselektron.



\begin{align}
    \left(-\frac{\hbar^2}{2m}\Vec{\nabla}^2 + V(r)\right)\psi &= E\psi
\end{align}

\begin{align}
    \Vec{\nabla}^2 &= \frac{1}{r} \frac{\partial}{\partial r}\left(r^2 \frac{\partial}{\partial r}\right) - \frac{1}{r^2}\Vec{l}^2
\end{align}

\begin{align}
    \Vec{l} &= \frac{1}{\hbar} \Vec{r} \cross \Vec{p} \: , \quad \text{hvor} \quad \Vec{p} = -i\hbar\Vec{\nabla}
\end{align}

\begin{align}
    \Vec{l}^2 &= -\left\{\frac{1}{\sin(\theta)}\frac{\partial}{\partial \theta}\left(\sin(\theta)\frac{\partial}{\partial \theta}\right) + \frac{1}{\sin^2(\theta)}\frac{\partial^2}{\partial \phi^2}\right\}
\end{align}

\begin{align}
    \psi &= R(r)Y(\theta,\phi)
\end{align}

\begin{align}
    &\frac{1}{R}\frac{\partial}{\partial r} \left(r^2 \frac{\partial R}{\partial r}\right) - \frac{2m_er^2}{\hbar^2}\left(V(r) - E\right) = \frac{1}{Y}\Vec{l}^2Y = b \\
    &\Rightarrow \begin{cases}
        \Vec{l}^2Y = bY \\
        -\frac{\hbar^2}{2m_e} \frac{\partial^2P}{\partial r^2} + \left(\frac{b}{2m_er^2} - \frac{e^2}{4\pi\epsilon_0 r}\right) = EP \: , \quad P = rR(r)
    \end{cases}
\end{align}