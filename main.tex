\documentclass[a4paper,12pt,amsmath,amssymb,aps]{article}
\usepackage[utf8]{inputenc}
\usepackage[danish]{babel}
\usepackage[T1]{fontenc}

% Fysik
\usepackage{physics}

% Matematik
\usepackage{amsmath,amssymb,bm,mathtools}
\usepackage{gensymb}
%\usepackage{breqn}
% Punktum i math-felter bliver til komma i pdf
%\DeclareMathSymbol{.}{\mathord}{letters}{"3B}

% SI-units
\usepackage[number-unit-product = \text{ }, inter-unit-product =\cdot]{siunitx}

% Figurer
\usepackage{graphicx}
\usepackage{float}

% Om dokumentet
\title{Dispositioner}
\author{Josefine Bjørndal Robl}
\date{\today}

% Marginer
\usepackage[tmargin=1.0in,bmargin=1in,lmargin=1.25in,rmargin=1.25in]{geometry}

% Header og footer
\usepackage{lastpage}
\usepackage{fancyhdr}
\pagestyle{fancy}
\lhead{
    Josefine Bjørndal Robl \\
	Aarhus Universitet - 201706760
}
\chead{}
\rhead{
	Atom- og Molekylefysik \\
	Eksamensdispositioner
}
\lfoot{}
\cfoot{}
\rfoot{Side \thepage\ af \pageref{LastPage}}
\renewcommand{\headrulewidth}{1pt}
\renewcommand{\footrulewidth}{1pt}

% Indryk tekst efter section og subsection
\usepackage{indentfirst}

% Ingen orddeling i section og subsection titler
\usepackage[raggedright]{titlesec}

% Ligninger nummereres 1.1, 1.2 osv. i section 1 og 2.1, 2.2 osv. i section 2
%\numberwithin{equation}{section}

% Ligninger nummereres 1.1.1, 1.1.2 osv. i subsection 1 og 2.1.1, 2.1.2 osv. i subsection 2 - dog også 1.0.1 og 2.0.1 i hhv. section 1 og 2.
%\numberwithin{equation}{subsection}

% Subsections med uden at hedde f.eks. 0.1 og 0.2 (uden sections)
%\renewcommand{\thesubsection}{\arabic{subsection}}

% Til at indsætte MATLAB kode i ShareLaTeX
% \usepackage{listings}
% \usepackage{color}
%     \definecolor{codegreen}{rgb}{0,0.6,0}
%     \definecolor{codegray}{rgb}{0.5,0.5,0.5}
%     \definecolor{codepurple}{rgb}{0.58,0,0.82}
%     \definecolor{backcolour}{rgb}{0.95,0.95,0.92}
% \lstdefinestyle{mystyle}{
%     backgroundcolor=\color{backcolour},
%     commentstyle=\color{codegreen},
%     keywordstyle=\color{magenta},
%     numberstyle=\tiny\color{codegray},
%     stringstyle=\color{codepurple},
%     basicstyle=\footnotesize,
%     breakatwhitespace=false,
%     breaklines=true,
%     captionpos=b,
%     keepspaces=true,
%     numbers=left,
%     numbersep=5pt,
%     showspaces=false,
%     showstringspaces=false,
%     showtabs=false,
%     tabsize=2
% }
% \lstset{style=mystyle}

% Lister med bindestreg som punkt
\renewcommand{\labelitemi}{\textendash}
\renewcommand{\labelitemii}{\textendash}
\renewcommand{\labelitemiii}{\textendash}
\renewcommand{\labelitemiv}{\textendash}

% Link/url - f.eks. så Table of content linker til section, subsection..., men også så url'er kan indsættes som clickable links.
\usepackage{hyperref}
\hypersetup{
    allcolors=black, % No colored links
    linktoc=all,     % "all" = both sections and subsections linked
}

% bra-ket notation
\usepackage{braket}
\renewcommand{\braket}{\Braket}

% Limit med limits under "lim"
\newcommand{\Lim}[2]{\raisebox{0.5ex}{\scalebox{0.8}{$\displaystyle \lim_{{#1} \rightarrow {#2}}\;$}}}
% Limit gående mod uendelig
\newcommand{\Liminf}[1]{\raisebox{0.5ex}{\scalebox{0.8}{$\displaystyle \lim_{{#1} \rightarrow \infty}\;$}}}
% Limit gående mod 0
\newcommand{\Limnul}[1]{\raisebox{0.5ex}{\scalebox{0.8}{$\displaystyle \lim_{{#1} \rightarrow 0}\;$}}}

% phi
\renewcommand{\phi}{\varphi}

% Til subfigures
\usepackage{caption}
\usepackage{subcaption}

%%%%%%%%%%%%%%%%%%%%%%%%%%%%%%%%%%%%%%%%%%%%%%%%%%%%%%%%%%%%%%%%%%%%%%%%%%

\begin{document}

\maketitle

% Giver mulighed for sidehoved og -fod på titelsiden
\thispagestyle{fancy}

% SKRIV HERUNDER
\setcounter{tocdepth}{1}
\tableofcontents

\section{Briefly outline Bohr's model of the atom. What are the main problems of this model?}

\huge

\normalsize
\subsection{Briefly outline Bohr's model of the atom. What are the main problems of this model?}

Bohrs atommodel bygger på den kendte viden fra Rutherfords eksperiment om, at atomer består af en lille og tung kerne med positiv ladning. Ser vi på hydrogen har vi en proton som kerne og en elektron kredsende omkring denne. Tiltrækningen mellem de to ladede partikler er givet ved Couloubkraften, hvilken er invers proportional med $r^2$ ligesom tyngdekraften er, hvorfor Bohrs atommodel bygger på den klassiske model for et tolegmesystem, f.eks. en planet omkring en stjerne, hvor banen er en cirkelbane, som fra klassiske mekanik opretholdes af en balance mellem centripetalkraften og Coulombbkraften
\begin{align} \label{eq:Q01_Centripetalkraft=Coulombkraft}
	\frac{m_e v^2}{r} &= F_\text{centripetal} = F_\text{Coulomb} = \frac{e^2}{4\pi\epsilon_0 r^2} \: ,
\end{align}
hvor $m_e$ er elektronens masse, $v$ er elektronens fart, $e$ er elementarladningen, hvilken er svarende til størrelsen af ladningen af en elektron eller proton, og $r$ er radius af banen.

Fra dette kan vinkelfrekvensen $\omega = v/r$ bestemmes
\begin{align}
    \frac{1}{r}\frac{m_e v^2}{r} &= \frac{m_e v^2}{r^2} = \frac{1}{r}\frac{e^2}{4\pi\epsilon_0 r^2} = \frac{e^2}{4\pi\epsilon_0 r^3} \nonumber\\
    \Rightarrow \omega^2 &= \frac{v^2}{r^2} = \frac{e^2}{4\pi\epsilon_0 m_e r^3} \: ,
\end{align}
hvilket er ækvivalent til Keplers lov, hvilket også var formodet, da der er benyttet de samme antagelser og atomet lige pt. er beskrevet fuldstændig klassisk.

Den totale energi i en sådan banebevægelse er givet ved summen af den kinetiske og potentielle energi
\begin{align}
    E &= T - U = \frac{1}{2}m_e v^2 - \frac{e^2}{4\pi\epsilon_0 r} \: ,
\end{align}
og ved at benytte \cref{eq:Q01_Centripetalkraft=Coulombkraft} fås den kinetiske energi til at være halvdelen af den potentielle energi
\begin{align}
    m_e v^2 &= \frac{e^2}{4\pi\epsilon_0 r}r = \frac{e^2}{4\pi\epsilon_0 r} \: ,
\end{align}
så den totale energi bliver
\begin{align} \label{eq:Q01_TotalenergiIHydrogenAtom}
    E &= - \frac{1}{2}\frac{e^2}{4\pi\epsilon_0 r} = - \frac{e^2}{8\pi\epsilon_0 r} \: ,
\end{align}
hvilken er negativ, da elektronen er i en bunden banebevægelse, så der skal tilføres energi til systemet for at fjerne elektronen fra banen.

Bohr formodede da, at kun baner med bestemte energier var tilladte, og at elektronen kun ændrer energi, når den hopper mellem baner, og denne overskudsenergi, når den hopper fra en yderlige bane til en inderligere bane, udsendes som lys af en bølgelængde bestemt ved energiforskellen mellem banerne.

\paragraph{Problem I -- Ingen stråling:} Bohr antager, at en ændring af energi kun kan finde sted, når elektronen hopper mellem de tilladte baner, men ud fra klassisk elektrodynamik udsender en ladet partikel i bevægelse energi i form af elektromagnetiske bølger (lys), men dette ville også betyde, at elektronen gradvist ville spiralere ind mod protonen og dermed også kunne forholde sig i mellemrummet mellem de tilladte baner. Dette stemmer ikke overens med Bohrs andre antagelser, hvorfor han antog -- da modellen ikke forklarer det -- at elektronen ikke radierer i dens banebevægelse, hvilket viser sig at holde stik med de eksperimentelle data.

\ldots
\begin{align}
    m_e v r &= n\hbar \: , \quad \forall n\in \mathbb{N} \: .
\end{align}
Kombineres denne med \cref{eq:Q01_Centripetalkraft=Coulombkraft} fås radius som værende
\begin{align} \label{eq:Q01_RadiusSomFunktionAfN}
    \frac{m_e v^2}{r} &= \frac{e^2}{4\pi\epsilon_0 r^2} \nonumber\\
    \Rightarrow \frac{1}{r} &= \frac{e^2}{4\pi\epsilon_0 m_e v^2 r^2} = \frac{e^2 m_e}{4\pi\epsilon_0 m_e^2 v^2 r^2} \nonumber\\
    \Rightarrow r &= \frac{4\pi\epsilon_0 m_e^2 v^2 r^2}{e^2 m_e} = \frac{4\pi\epsilon_0 \hbar^2 n^2}{e^2 m_e} = \frac{\hbar^2 n^2}{\left(e^2/4\pi\epsilon_0\right) m_e} = a_0 n^2 \: ,
\end{align}
hvor $a_0$ er Bohrradien,
\begin{align}
    a_0 &= \frac{\hbar^2}{\left(e^2/4\pi\epsilon_0\right) m_e} \: .
\end{align}
Fra \cref{eq:Q01_TotalenergiIHydrogenAtom} og \cref{eq:Q01_RadiusSomFunktionAfN} kan man få Bohr formlen
\begin{align} \label{eq:Q01_BohrsFormel}
    E &= - \frac{e^2}{8\pi\epsilon_0 r} = - \frac{e^2}{8\pi\epsilon_0 a_0} \frac{1}{n^2} \: ,
\end{align}
hvor $n$ kaldes \textsf{hovedkvantetallet} (eng. principal quantum number).
Bohrs formel forudsiger, at lyset udsendt fra overgange, $n' \rightarrow n$, mellem de tilladte baner i atomet vil have bølgetal ($\tilde{\nu} = 1/\lambda$)
\begin{align} \label{eq:Q01_BoelgetalFraBohrsFormel}
    \tilde{\nu} &= R_\infty \left(\frac{1}{n^2} - \frac{1}{n'^2}\right) \: .
\end{align}
Udregnes de mulige bølgetal for hydrogen vil disse stemme forholdsvis overens med de udregnede bølgetal fra Rydbergs formel,
\begin{align}
    \tilde{\nu} &= \frac{1}{\lambda} = R \left(\frac{1}{n^2} - \frac{1}{n'^2}\right) \: .
\end{align}
Rydbergkonstanten $R_\infty$ i \cref{eq:Q01_BoelgetalFraBohrsFormel} er defineret som
\begin{align}
    hcR_\infty &= \frac{\left(e^2/4\pi\epsilon_0\right)^2 m_e}{2\hbar^2} \: ,
\end{align}
hvor $hc$ er en konversionsfaktor mellem energi og kvantetal, siden $R_\infty$ er givet i enheder af $\si{\per\metre}$ (eller $\si{\per\centi\metre}$, hvilket normalt benyttes). I laboratoriet har man ved hjælp af lasere kunnet finde en utrolig præcis værdi for Rydbergkonstanten $R_\infty = \SI{10973731.568525}{\per\metre}$.

\paragraph{Problem II -- Antagelse om uendelig tung kerne:} \ldots





Modellen
- Kvantiseret banebevægelse => Modellen kunne forklare de observerede spektre for hydrogen.
- Postulaterne
-- 1. Orbits as in classical mechanics
-- 2. Only particular orbits allowed
-- 3. Electron jumps between orbits
-- 4. Light energy h∙f = ΔE.

Problemer
- Eldyn: Ladede partikler i bevægelse udsender elektromagnetiske bølger. Bohr: Jeg antager, at de ikke udsender elektromagnetiske bølger. Data: Stemmer overens med Bohrs model.
-- Bohrs model fortæller ikke hvorfor, at de ikke udsender dem, men antager blot, at atomer ikke gør.
- Antager uendelig tung kerne, aka. stillestående kerne i centrum og kun bevægende elektroner i cirkelbaner.
-- Der skal tages højde for reduceret masse, hvilken er forskellig mellem atomers isotoper. Dette leder til en lille men aflæselig observerbar forskel i frekvensen af det udsendte lys fra forskellige isotoper i samme atom, hvilket kaldes isotopskift.

\section{Describe interaction of atoms with light by using the Einstein coefficients. Establish relationships between these coefficients. Relate the Einstein coefficients to spectroscopic experiments.}

\huge

\normalsize
\subsection{Tekst}

\emph{Describe interaction of atoms with light by using the Einstein coefficients. Establish relationships between these coefficients. Relate the Einstein coefficients to spectroscopic experiments.}

Noter
- S. 11-13
- PP forelæsning 2

To-level atom med energiniveauer E1 og E2. Antal elektroner i hver tilstand, hhv. N1 og N2.
Lys med energitæthed rho(omega) skinner ind på atom => Overgang fra nedre til øvre level med rate proportionalt til rho(omega12), hvor B12 er proportionalitetskonstant. Atomet interagerer meget kun med den del af lyset, som har en frekvens lige omkring omega12 = (E1 - E2)/hbar, hvilket er atomets resonansfrekvens.
Per symmetri formodes det også, at der skabes overgang fra øvre til nedre level med en rate proportional med energidensiteten med proportionalitetskonstant B21.
"This (above) is a process of stimulated emission in which the radiation
at angular frequency ω causes the atom to emit radiation of the same
frequency."
The symmetry between up and down is broken by the process of spontaneous emission in which an atom falls down to the lower level, even when no external radiation is present. Einstein introduced the coefficient A21 to represent the rate of this process. Thus the rate equations for the populations of the levels, N1 and N2, are eq. (1.25) and (1.26).
Når pho(omega) = 0 og N2 =/= 0 falder populationen i øvre level eksponentielt eq. (1.27), hvor levetiden er 1/A21

Place the atom in a region of black-body radiation: (Equations of page 13)

\section{Briefly outline the main result of time-dependent perturbation theory in a two level system. Explain how this result is related to spectroscopic experiments.}

\huge

\normalsize
\subsection{Tekst}

\emph{Briefly outline the main result of time-dependent perturbation theory in a two level system. Explain how this result is related to spectroscopic experiments.}

\section{Discuss the coherent interaction between atoms and light using the Bloch vector formalism. Give an example for experiment which can be described using the Bloch vector.}

\huge

\normalsize
\subsection{Tekst}

\emph{Discuss the coherent interaction between atoms and light using the Bloch vector formalism. Give an example for experiment which can be described using the Bloch vector.}

\section{Briefly outline the solution of the Schrödinger equation for hydrogen and comment on the result. What are the separation constants and why is the energy degenerate in $l$ and $m_l$?}

\huge

\normalsize
\subsection{Tekst}

\emph{Briefly outline the solution of the Schrödinger equation for hydrogen and comment on the result. What are the separation constants and why is the energy degenerate in $l$ and $m_l$?}

\section{Discuss how the dipole approximation for the atom light interaction leads to the transition dipole moment and outline how this leads to the selection rules.}

\huge

\normalsize
\subsection{Tekst}

\emph{Discuss how the dipole approximation for the atom light interaction leads to the transition dipole moment and outline how this leads to the selection rules.}

\section{Outline how the selection rules for $m$, $l$, and $s$ were obtained and explain them based on the angular momentum of the photon. Under what condition are these selection rules valid?}

\huge

\normalsize
\subsection{Tekst}

\emph{Outline how the selection rules for $m$, $l$, and $s$ were obtained and explain them based on the angular momentum of the photon. Under what condition are these selection rules valid?}

\section{Describe how the Stern-Gerlach experiment led to the discovery of the spin of the electron. Which properties does the spin possess?}

\huge

\normalsize
\subsection{Tekst}

\emph{Describe how the Stern-Gerlach experiment led to the discovery of the spin of the electron. Which properties does the spin possess?}

\section{The electron and the nucleus are assumed to possess an angular momentum. What consequences do these angular momenta have?}

\huge

\normalsize
\subsection{Tekst}

\emph{The electron and the nucleus are assumed to possess an angular momentum. What consequences do these angular momenta have?}

\section{Outline how the spin-orbit coupling arises. Discuss the resulting splitting of the energy states for the case of hydrogen.}

\huge

\normalsize
\subsection{Tekst}

\emph{Outline how the spin-orbit coupling arises. Discuss the resulting splitting of the energy states for the case of hydrogen.}

\section{Discuss the symmetry properties of the He wave functions and show which solutions are possible due to the Pauli principle.}

\huge

\normalsize
\subsection{Tekst}

\emph{Discuss the symmetry properties of the He wave functions and show which solutions are possible due to the Pauli principle.}

\section{Describe the He spectrum (singlet and triplet states) and explain why it was initially believed that there are two kinds of helium (called para- and orthohelium).}

\huge

\normalsize
\subsection{Tekst}

\emph{Describe the He spectrum (singlet and triplet states) and explain why it was initially believed that there are two kinds of helium (called para- and orthohelium).}

\section{Discuss the ground states of larger atoms (up to Ne) by using Hund´s first rule. Explain the general structure of the periodic table of elements (e.g. transition metals) based on the population of the electronic shells.}

\huge

\normalsize
\subsection{Tekst}

\emph{Discuss the ground states of larger atoms (up to Ne) by using Hund´s first rule. Explain the general structure of the periodic table of elements (e.g. transition metals) based on the population of the electronic shells.}

\section{Explain the term symbol (assuming L-S coupling) which describes the state of a multi-electron atom. Use an alkali atom to give an example.}

\huge

\normalsize
\subsection{Tekst}

\emph{Explain the term symbol (assuming L-S coupling) which describes the state of a multi-electron atom. Use an alkali atom to give an example.}

\section{How does an external magnetic field affect an atom in LS-coupling? Describe the Zeeman effect without accounting for the nucleus.}

\huge

\normalsize
\subsection{Tekst}

\emph{How does an external magnetic field affect an atom in LS-coupling? Describe the Zeeman effect without accounting for the nucleus.}

\section{Outline the origin of the hyperfine structure. What are the assumptions made for the nucleus? Discuss the resulting splitting of the energy states.}

\huge

\normalsize
\subsection{Tekst}

\emph{Outline the origin of the hyperfine structure. What are the assumptions made for the nucleus? Discuss the resulting splitting of the energy states.}

\section{Discuss the “Zeeman Effect” in the presence of hyperfine structure.}

\huge

\normalsize
\subsection{Tekst}

\emph{Discuss the “Zeeman Effect” in the presence of hyperfine structure.}

\section{Outline how spontaneous emission can be included in the semiclassical description of atom-light interaction and discuss the result using the optical Bloch vector.}

\huge

\normalsize
\subsection{Tekst}

\emph{Outline how spontaneous emission can be included in the semiclassical description of atom-light interaction and discuss the result using the optical Bloch vector.}

\section{Explain different physical mechanisms, which can lead to molecular binding between atoms. Which type of molecular potentials can be expected? Discuss approximations to these potentials.}

\huge

\normalsize
\subsection{Tekst}

\emph{Explain different physical mechanisms, which can lead to molecular binding between atoms. Which type of molecular potentials can be expected? Discuss approximations to these potentials.}

\section{Explain the origin of the molecular potential within the Born-Oppenheim approximation and discuss the vibrational and rotational structure of diatomic molecules.}

\huge

\normalsize
\subsection{Tekst}

\emph{Explain the origin of the molecular potential within the Born-Oppenheim approximation and discuss the vibrational and rotational structure of diatomic molecules.}


\end{document}

%%%%%%%%%%%%%%%%%%%%%%%%%%%%%%%%%%%%%%%%%%%%%%%%%%%%%%%%%%%%%%%%%%%%%%%%%%

% HJÆLPENDE KOMMANDOER

% FIGURER
    % Figurnavn "?.jpg" skal være i ét ord eller med "_" (underscore) som mellemrum
\begin{figure}[!ht]
    \centering
    \includegraphics[scale=0.4]{?.jpg}
    \caption{Billedtekst}
    \label{fig:figur X}
\end{figure}


% LIGNINGER
    % ALIGN UNDER HINANDEN MED "&"
        % Det der står efter "&" stilles under hinanden
    \begin{align}
    	K_B+U_B&=K_A+U_A+W_{andet} \\
    	\Leftrightarrow U_A&=K_B+U_B-K_A-W_{andet}
    \end{align}
    
    % Matricer i align
    \begin{align}
        \begin{bmatrix}
            
        \end{bmatrix}  
    \end{align}
    
    % SUBEQUATIONS OG SPLIT
        % "Subequation" giver mulighed for at undernummerere delligninger som f.eks. 1a, 1b, 1c osv..
        % "Split" giver herudover mulighed for kun ikke at undernummerere hver eneste af ligningerne, som f.eks. i at man har to ligninger, hvor den ene er en udledning, men den kun skal have ét nummer, selvom den dækker flere linjer.
    \begin{subequations}
    \begin{align}
    \begin{split} \label{eq:lign. Xa}
    	K_B+U_B&=K_A+U_A+W_{andet} \\
    	\Leftrightarrow U_A&=K_B+U_B-K_A-W_{andet}
    \end{split}
    \\
    \begin{split} \label{eq:lign. Xb}
    	\Leftrightarrow U&=mgy
    \end{split}
    \end{align}
    \end{subequations}
    
    % BREAK TIL NÆSTE LINJE VED "\\"
    \begin{align}
    	K_B+U_B&=K_A+U_A+W_{andet} \\
    	\Leftrightarrow U_A&=K_B+U_B-K_A-W_{andet}
    \end{align}
    
    % UNDGÅ NUMMERERING AF BESTEMTE LINJER I LIGNINGEN VED "\nonumber"
    \begin{align}
    	K_B+U_B&=K_A+U_A+W_{andet} \nonumber
    \end{align}
    
    % LABEL
    \begin{align} \label{eq:lign. X}
    	K_B+U_B&=K_A+U_A+W_{andet} \\
    	\Leftrightarrow U_A&=K_B+U_B-K_A-W_{andet}
    \end{align}